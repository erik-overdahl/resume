\documentclass{article}

\usepackage[hmargin=2cm, vmargin=1cm]{geometry}
\usepackage[hidelinks]{hyperref}
\usepackage{enumitem}

\pagestyle{empty}

% Section formatting
\usepackage{titlesec}
% Usage:
% \titleformat{command}[shape]{format}{label}{sep}{before-code}[after-code]
% command one of: \part, \chapter, \section, \subsection,
%                 \subsubsection, \paragraph, \subparagraph
% shape one of:   hang, block, display, runin, leftmargin,
%                 rightmargin, drop, wrap, frame
\titleformat{\section}
[display]
{\bfseries\Large}
{}
{0.5ex}
{}
[\titlerule]

\newcommand{\resumeSection}[1]{\section*{#1}}
\newcommand{\institution}[1]{\subsection*{\scshape{#1}}}
\newcommand{\jobPosition}[3]{\subsubsection*{\scshape{#1}\hfill #2 -- #3}}

\begin{document}

\begin{center}
  {\Huge\scshape{Erik Overdahl}}
  \\\vspace{5pt}
  \normalsize{
    St.\ Paul, MN
    $|$
    erik.overdahl@gmail.com
    $|$
    +1 (260) 437-0551
    \\
    \href{https://github.com/erik-overdahl}{GitHub: \underline{erik-overdahl}}
    $|$
    \href{https://linkedin.com/in/erik-overdahl}{LinkedIn: \underline{erik-overdahl}}
  }
\\\vspace{5pt}
  {\large\emph{
      Software Engineer with 4 Years Experience, Passionate About
      Performance and Tooling
    }
  }
\end{center}

\resumeSection{Work Experience}

  \institution{WiseTech Global}

    \jobPosition{DevOps Engineer}{Nov 2021}{Oct 2022}
    \begin{itemize}[noitemsep]
      \item
            Automated all infrastructure deployment by writing Ansible
            roles: building virtual machines in VMWare VSphere; provisioning app
            servers; configuring load balancers; setting up an
            Elasticsearch-Logstash-Kibana monitoring stack; establishing Redis
            instances for per-service and shared caching; and managing Jenkins.
      \item
            Maintained a complex ETL data pipeline capable of handling
            millions of events per minute using Kafka, ElasticSearch, Google BigQuery,
            and Apache Airflow.
      \item
            Automated pulling security vulnerability data from Tenable and
            applying patches to servers using Ansible and Python.
      \item
            Set up Nagios monitoring checks for critical systems, e.g.\ health of disaster recovery database failovers
    \end{itemize}

    \jobPosition{Software Developer}{July 2019}{Nov 2021}
    \begin{itemize}[noitemsep]
      \item
            Led development and testing for text extraction tool for
            shipping industry, including extensive development with
            both relational (Postgres) and graph (Neo4j) databases, which reduced time to
            export of millions of end-to-end freight shipping rates by
            a factor of 14.
      \item
            Built microservices responsible for processing customer
            data using Java Spring Boot.
    \end{itemize}

  \institution{UNC Eshelman School of Pharmacy Molecular Modeling Lab}

    \jobPosition{Research Assistant}{June 2018}{July 2019}
    \begin{itemize}[noitemsep]
      \item
            Collaborated with computational chemists to develop
            pipelines for text extraction and pharmaceutical machine
            learning models for predicting chemical activity using Python.
      \item
            Extracted and standardized toxicological data from more
            than 50,000 unstructured European Chemical Agency reports
            for use in model training.
      \item
            Developed several unique data cleaning pipelines that
            collected and collated data across thousands of diverse
            databases using Python, SQL, and web scraping.
    \end{itemize}

\resumeSection{Education}
  \institution{Bradfield School of Computer Science}
    \jobPosition{Computer Science Intensive}{Aug 2021}{Aug 2022}

    \emph{Projects}
    \begin{itemize}[noitemsep]
      \item A full key-value store with custom SSTable file format
      \item A distributed key-value store
      \item Implementation of a reliable data transfer protocol built on top of UDP
      \item Built a DNS client and an HTTP proxy server using Unix sockets
      \item Improved performance of compute-heavy code by improving cache locality
    \end{itemize}

  \institution{St.\ Olaf College}
    \jobPosition{BA in Economics and Mathematics with Statistics concentration}{2014}{2018}

\resumeSection{Skills/Technologies}
  \begin{description}
          \item [Languages]
    Java,
    Go,
    Python,
    Bash,
    SQL,
    Cypher
          \item [Tools]
    Unix,
    Docker,
    Git,
    Postgres,
    Neo4j,
    Redis,
    Ansible,
    Nagios,
    Apache Kafka,
    Avro,
    Elasticsearch,
    Logstash
  \end{description}

\end{document}
